\documentclass[a4paper, 12pt]{article}
\usepackage[utf8]{inputenc}
\usepackage[brazilian]{babel}
\usepackage{indentfirst}
\usepackage[top=1in, bottom=1in]{geometry}

\title{Análise de complexidade do Algoritmo de Floyd Modificado}
\author{Matheus Silva Pinheiro Bittencourt}
\date{\today}

\begin{document}
\maketitle

\section*{Algoritmo}
\begin{verbatim}
FloydModificado ()
início
    para i = 1 até n faça
        para j = 1 até n faça
            A[i,j] <- D[i,j];
            R[i,j] <- 0;
        fim para
    fim para

    para i = 1 até n faça
        A[i,i] <- 0;
    fim para

    para k = 1 até n faça
        para i = 1 até n faça
            para j = 1 até n faça
                se A[i,k] + A[k,j] < A[i,j] então
                    A[i,j] <- A[i,k] + A[k,j];
                    R[i,j] <- k;
            fim para
        fim para
    fim para
fim
\end{verbatim}

\section*{Análise}

No primeiro \verb|for|, analisando de dentro para fora:
\begin{itemize}
    \item As duas operações de atribuição custarão $2n$, o \verb|for| terá que
    fazer $n$ incrementos e $n+1$ testes, e mais $1$ da atribuição
    \verb|j = 1|. Logo $T_{interno}(n) = 2n + n + n + 1 + 1 = 4n + 2$
    \item No for externo, serão feitos $n$ incrementos, $n+1$ testes, mais $1$
    de custo para a atribuição de \verb|i|, isso adicionado a $n*T_{interno}(n)
    $. Logo:

    \begin{math}
    T_{externo}(n) = n + n + 1 + 1 + n*T_{interno}(n) \\
    T_{externo}(n) = 2n + 2 + 4n^2 + 2n \\
    T_{externo}(n) = \mathbf{T_{for1}(n) = 4n^2 + 4n + 2}
    \end{math}
\end{itemize}

No segundo \verb|for|, o custo do \verb|for| será $n$, pois só é feito uma
atribuição dentro do loop, também terá $n+1$ de custo dos testes, $1$ da
atribuição de \verb|i|, e $n$ dos incrementos de \verb|i|. Logo, para esse
\verb|for|, $\mathbf{T_{for2}(n) = 3n + 2}$.

\mbox{}

Para o terceiro \verb|for|, sabemos que cada \verb|for| custa $2n + 2 + n*T(n)$
onde $T(n)$ é o custo de execução da parte interna do \verb|for|. Expandindo
essa ``fórmula'' temos:

\mbox{}

\noindent
\begin{math}
T_{for3}(n) = 2n + 2 + n(2n + 2 + n(2n + 2 + n*5))\\
T_{for3}(n) = 2n + 2 + n(2n + 2 + 2n^2 + 2n + 5n^2)\\
T_{for3}(n) = 2n + 2 + n(7n^2 + 4n + 2)\\
T_{for3}(n) = 2n + 2 + 7n^3 + 4n^2 + 2n\\
\mathbf{T_{for3}(n) = 7n^3 + 4n^2 + 4n + 2}
\end{math}

\mbox{}

No final $T(n) = 5$ pois no \verb|for| interno são feitas duas somas, duas
atribuições, e um teste, logo, para esse \verb|for| temos $T(n) = 2n + 2 + 5n$,
terminando essa espécie de chamada recursiva a $T(n)$.

\section*{Conclusão}

Somando todos os T(n) temos:

\mbox{}

\noindent
\begin{math}
T_{total}(n) = T_{for1}(n) + T_{for2}(n) + T_{for3}(n)\\
T_{total}(n) = 4n^2 + 4n + 2 + 3n + 2 + 7n^3 + 4n^2 + 4n + 2\\
T_{total}(n) = 7n^3 + 8n^2 + 11n + 6
\end{math}

\mbox{}

\noindent
Portanto a complexidade assintótica do algoritmo é $\mathcal{O}(n^3)$.

\end{document}
